\begin{resumo}
A disciplina Projeto Integrador 1, da Faculdade do Gama - Universidade de Brasília,
tem como objetivo promover a integração entre as diversas engenharias presentes no campus
num projeto comum. Dessa forma, possibilita que os alunos trabalhem diversas habilidades
altamente necessárias, como o trabalho em grupo e o capacidade de liderar grupos.

Este documento visa apresentar o projeto Prédio Sustentável e Inteligente,
realizado pela Turma A. O foco é sanar problemas enfrentados pelo corpo discente
e docente da FGA, como a iluminação má planejada, sistema arcaico de chamada,
além de desperdícios energéticos e hídricos.

O Prédio Sustentável e Inteligente consiste em utilizar fontes de energia e
materiais sustentáveis em seu desenvolvimento. Por meio de softwares e
sistemas eletrônicos pretende-se monitorar desperdícios hídricos e energéticos
construindo assim uma sistema Smart Grid no campus. Ademais, criar-se-á um
sistema de controle de acesso que monitore o fluxo de pessoas, e um sistema de
controle de presença mais moderno.

 \vspace{\onelineskip}

 \noindent
 \textbf{Palavras-chave}: Prédio Sustentável, Faculdade do Gama, Smart Grid.
\end{resumo}
