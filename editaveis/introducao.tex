\part{Aspectos Gerais}
\chapter[Introdução]{Introdução}
Controle, uma palavra chave para que um bom projeto seja desenvolvido. Um
projeto consiste em um processo único que tem como objetivo um produto inovador
de um grupo de afazeres sincronizados e gerenciados em um determinado período de
tempo. A disciplina Projeto integrador 1 busca por meio dessas características
mostrar como deve ser a realização do projeto.

Mediante os atributos a turma A irá elaborar o projeto de um Prédio sustentável
e inteligente com o objetivo de trazer conforto para todos os que utilizarem da
Faculdade do Gama, o projeto consiste em sanar problemas já enfrentados pelo
corpo discente e docente da universidade como por exemplo iluminação má
planejada, sistema de chamada em papel o que causa uma grande perca de tempo por
parte do professor, desperdícios energéticos e hídricos no campus também serão
solucionados no projeto.

O Prédio sustentável inteligente consiste em utilizar fontes de energia e
materiais sustentáveis, visto que atualmente a sustentabilidade é um ponto
crucial para um bom desenvolvimento da sociedade. Por meio de softwares e
sistemas eletrônicos pretende-se monitorar desperdícios hídricos e energéticos
construindo assim uma sistema Smart Grid na universidade além de criar um
sistema de controle e acesso que evite fraude e que o professor não perca tempo
para executá-lo.

Assim, houve uma busca para estabelecer os riscos e elicitar meios para
evitá-los e mitigá-los. Ademais, estruturou-se o projeto buscando estabelecer
equipamentos, custos e meios mais aprofundados, conclusivos e detalhados de
atender às demandas geradas pelos requisitos.

Dessa forma, com as engenharias do campus trabalhando de forma harmônica tem-se
como objetivo melhorar o ambiente de trabalho e de estudos para os que
frequentam o campus da FGA , fazendo assim que os professores e os alunos tenham
ferramentas e ambientes favoráveis para realizar uma boa didática e boas
pesquisas.


\chapter{Escopo}
Para a realização do projeto Prédio inteligente deve-se definir as especificações do limite que contempla o projeto, para isso foi necessário definir os requisitos e analisá-los e aplicar alguns métodos para facilitar a definição do escopo como o dos 5W e 2H representado abaixo:

1-What -O que será feito ?

2-Who- Por quem será feito ?

3-Where- Onde será feito ?

4-When-Quando será feito?

5-Why-Por que será feito?

6-How-Como será feito ?

7-How Much- Quanto custará?

Dessa forma o escopo foi definido como sendo o Prédio Sustentável inteligente da Faculdade do Gama(FGA) e pode ser melhor observado na EAP visto que a mesma guia a equipe para o término do projeto,não incluindo geração de energia ou algum sistema tecnológico para outra parte do campus exceto o estacionamento parte fundamental do campus que ainda não foi concluída e pode ser resolvida com ideias que possuem nesse projeto , esse escopo foi decidido com base: no  alto custo financeiro para suprir toda demanda energética do campus, futuras alterações de estruturas e dimensões dos prédios já construídos e pelo tempo de entrega do projeto.Por fim,foi realizado um escopo para que futuramente a Universidade de Brasília pudesse aproveitá-lo para a construção do novo prédio trazendo assim conforto e ferramentas necessárias para todos que necessitam do mesmo para absorver conhecimento (stakeholders).

Outra subdivisão do nosso projeto, trata da parte de controle de acesso, que é responsável por monitorar todo o acesso de pessoas feito neste nosso prédio tecnológico dentro da FGA. Levando em conta que o nosso trabalho é feito em uma universidade federal, não podemos fechar completamente o acesso à universidade, entretanto nosso objetivo é selecionar quem terá ou não acesso a partes específicas dentro do novo prédio.

Inicialmente a ideia é restringir uma parte do estacionamento só para alunos matriculados na faculdade, e para que seja liberada a entrada, é necessário a liberação automática mediante apresentação de carteirinha, em seguida, o acesso também será restringido nas salas e laboratórios com trancas que serão acopladas nas portas, no caso de laboratórios e salas com equipamentos mais sofisticados, somente será autorizada a entrada com um professor como responsável pelo uso do ambiente, e no caso das salas simples, também haverá necessidade de um responsável, contudo, poderá ser tanto aluno, monitor ou professor. A frequência também será averiguada mediante carteirinha em um aparelho eletrônico portátil que cada professor possuirá. Essa tecnologia e modo de segurança já é utilizado em várias universidades do mundo e especialmente em algumas faculdades particulares em Brasília e por isso é uma ideia a ser implementada dentro da UnB principalmente por questão de segurança.

Na parte estrutural do projeto serão aplicados materiais inteligentes e sustentáveis para amenizar a produção de resíduos e consequentemente o impacto no meio ambiente. Além disso, serão consideradas algumas adaptações sobre o uso das salas para se possa receber o sistema de automação e também quais serão destinadas a laboratórios ou salas de aula, de acordo com as necessidades dos usuários. Por fim, haverá a alteração da posição dos elementos usados em sala para melhorar o impacto que os mesmos têm no aprendizado.

Para a produção energética foi considerada duas formas para suprir a demanda energética da FGA a primeira e principal é a geração de energia por meio de placas fotovoltaicas e a segunda sendo utilizada como reserva será por meio de um gerador movido a biodiesel.

Tendo que até 7 mil pessoas, aproximadamente, podem utilizar o prédio, número de pessoas que também é a capacidade máxima do prédio, temos que o controle de acesso ao prédio custará em média 18.459,44 reais, considerando o preço de carteirinhas e aparelhos de controle de acesso nas salas e no estacionamento. Os custos relacionando a interface e processamento de dados estão atrelados a quanto custará o desenvolvimento do aplicativo e a incrementação no banco de dados para armazenar e processar os novos dados, com isso temos que em média será gasto 70.000,00 reais.

A instrumentação e controle envolve aparelhos e sensores que vão identificar condições do prédio e realizar ambientação nas salas custando aproximadamente 63179,00 reais. As estruturas e materiais do prédio custarão por volta de 1041797,10 de reais.

O controle energético do prédio é feito de forma inteligente e esse controle e as diferentes formas de geração de alternativas de energia custará aproximadamente 1.000.000,00 de reais, totalizando para todo prédio um valor próximo de 3.000.000,00 (3 milhões) de reais.

\chapter{Requisitos\label{ch:requisitos}}
\section{Backlog do Produto}
Com base nas técnicas apresentadas no Ponto de Controle 01 (Brainstorming, Entrevista), levantou-se os requisitos do projeto.

Estes requisitos foram documentados no Backlog do Produto, mostrado na figura ~\ref{fig:backlog}, onde os requisitos foram agrupados visando a rastreabilidade vertical, partindo de grandes blocos mais genéricos chamados Épicos, que são divididos em partes menores chamadas Features.

\begin{figure}[!h]
  \centering
  	\includegraphics[width=0.9\textwidth]{figuras/backlog.eps}
   \caption{Backlog do Produto\label{fig:backlog}}
\end{figure}

Os requisitos estão explicitados detalhadamente no apêndice \ref{appendix:apdcA}.

\chapter{Estudo de Riscos}
\section{Introdução}
Segundo o \cite{pmbok}  , a Gerência de Riscos de Projeto inclui os processos de planejamento, identificação,
análise, planejamento de respostas e controle de riscos de um projeto. Essa gerência tem como objetivos aumentar a
probabilidade e o impacto dos eventos positivos, e reduzir a probabilidade e o impacto dos eventos negativos no projeto.

\section{Identificação dos Riscos}
Para a identificação dos riscos foi utilizada a técnica SWOT. Nesta técnica são utilizados 4 campos: Força (\textit{Strengths}),
 Oportunidades (\textit{Opportunities}), Fraqueza (\textit{Weaknesses}) e Ameaças (\textit{Threats}). Os campos Força e Oportunidade permitem uma
 visualização amplificada do projeto ao passo que as Fraquezas e Ameaças devem ser devidamente tratadas por meio do
 gerenciamento de riscos para manter o projeto seguindo corretamente. Os campos são definidos como:

\begin{itemize}
  \item \textbf{Força (\textit{Strengths}):} São elementos internos que representam benefícios para o projeto.
  \item \textbf{Oportunidades (\textit{Opportunities}):} São elementos externos que representam benefícios para o projeto.
  \item \textbf{Fraqueza (\textit{Weaknesses}):} São elementos internos que trazem prejuízos para o projeto.
  \item \textbf{Ameaças (\textit{Threats}):} São elementos externos que trazem prejuízos para o projeto.
\end{itemize}

Segue-se a representação gráfica desta técnica aplicada à este projeto.

\pagebreak

\begin{figure}[!h]
 \centering
 \includegraphics[keepaspectratio=true,scale=0.23]{figuras/swot.eps}
 \caption{Técnica SWOT}
\end{figure}

\section{Categorização dos Riscos}
Risco é um evento ou uma condição incerta que, se ocorrer, provocará um efeito positivo ou negativo nos objetivos do projeto tais como custo, escopo, prazo ou qualidade \cite{bianco}. A Estrutura Analítica de Riscos permite uma organização onde há maior compreensão, gerenciamento e comunicação pois apresenta seus resultados de forma estruturada como mostrado abaixo.
\subsection{Descrição dos Itens da EAR}
\begin{itemize}
  \item Técnico
    \begin{itemize}
      \item Inexperiência em construção civil: Diz respeito a falta de familiaridade do grupo com técnicas pautadas na engenharia civil.
    \end{itemize}
  \item Externo
  \begin{itemize}
    \item \textbf{Mudanças no calendário acadêmico:} Diz respeito ao ambiente acadêmico onde está inserido o projeto e os riscos de mudança no calendário.
    \item \textbf{Ausência de informações quanto a documentação a ser elaborada:} Diz respeito a falta de informações na disciplina quanto ao que deve ser feito em que etapa.
    \item \textbf{Crise econômica:} Diz respeito a instabilidade econômica pela qual vem passando o país nos últimos anos.
    \item \textbf{Resultado final não atender as expectativas:} Diz respeito a não aceitação do projeto por parte do cliente e stakeholders.
  \end{itemize}


  \item Organizacional
  \begin{itemize}
    \item \textbf{Atraso na entrega das atividades:} Diz respeito a desorganização da equipe quanto a respeitar o calendário.
    \item \textbf{Redução dos membros:} Diz respeito aos riscos relacionados aos recursos humanos do projeto, com membros que saem da equipe.
    \item \textbf{Falta de comprometimento dos membros:} Diz respeito a problemas de gerenciamento e cobrança para com membros ociosos.
    \item \textbf{Erros de estimativa e previsões:} Diz respeito a falta de experiência da equipe quanto a gerência de projetos que pode acarretar em atrasos ou adiantamentos.
  \end{itemize}

  \item Gerência
  \begin{itemize}
    \item \textbf{Inexperiência em projetos grandes:} Diz respeito a imaturidade da equipe ao lidar com uma equipe grande.
  \end{itemize}

\end{itemize}

\pagebreak

\section{Registro de Riscos Identificados}
\subsection{Riscos Negativos}
\begin{table}[h]
  \centering
  \caption{Registro dos Riscos Negativos}
  \begin{tabular}{|l|c|l|l|}
    \hline
    \multicolumn{4}{|c|}{\textbf{Riscos Negativos}}                                                                                                                                                                                                               \\ \hline
    \multicolumn{1}{|c|}{\textbf{Evento}}                          & \textbf{Identificador} & \multicolumn{1}{c|}{\textbf{Impacto}}                                    & \multicolumn{1}{c|}{\textbf{Causa}}                                                      \\ \hline
    \parbox[t]{5cm}{Mudança do calendário acadêmico.}                               & R01                    & \parbox[t]{4cm}{Adiantamento ou atrazo das entregas}                                      & \parbox[t]{4cm}{Greves ou qualquer outro empecilho relacionado a universidade}                            \\ \hline
    Redução dos membros.                                           & R02                    & \parbox[t]{4cm}{Sobrecarga de trabalho para os outros membros do grupo}                   & \parbox[t]{4cm}{Trancamento ou desistência da diciplina}                                                  \\ \hline
    \parbox[t]{5cm}{Crise econômica no país pode tornar o projeto inviável.}        & R03                    & \parbox[t]{4cm}{Projeto não será utilizado}                                               & \parbox[t]{4cm}{Atual instabilidade econômica do país}                                                    \\ \hline
    \parbox[t]{5cm}{Falta de comprometimento dos membros.}                          & R04                    & \parbox[t]{4cm}{Sobrecarga de trabalho para os outros membros do grupo}                   & \parbox[t]{4cm}{Membros querem desistir da diciplina}                                                     \\ \hline
    \parbox[t]{5cm}{Erros de estimativas e previsões.}                              & R05                    & \parbox[t]{4cm}{Previsões e estimativas muito foras da realidade do prédio}               & \parbox[t]{4cm}{Inexperiência nas atividades de estimativa e previsões}                                   \\ \hline
    \parbox[t]{5cm}{Resultado final não atender às expectativas.}                   & R06                    & \parbox[t]{4cm}{Rejeição do projeto}                                                      & \parbox[t]{4cm}{Falhas nas especificações e validações dos requisitos}                                    \\ \hline
    \parbox[t]{5cm}{Atraso nas entregas das atividades.}                            & R07                    & \parbox[t]{4cm}{Atraso no cronograma}                                                     & \parbox[t]{4cm}{Falta de motivação experiência ou planejamento da equipe}                                 \\ \hline
    \parbox[t]{5cm}{Inexperiência da equipe em projetos grandes.}                   & R08                    & \parbox[t]{4cm}{Problemas de gerência de projeto}                                         & \parbox[t]{4cm}{Membros nunca trabalharam em um grande projeto}                                           \\ \hline
    \parbox[t]{5cm}{Ausência de informações quanto à documentação a ser elaborada.} & R09                    & \parbox[t]{4cm}{Documentação entregue incompleta}                                         & \parbox[t]{4cm}{Falta de fontes de informação sobre a necessidade de documentos e informações no projeto} \\ \hline
    \parbox[t]{5cm}{Inexperiência quanto à construção civil.}                       & R10                    & \parbox[t]{4cm}{Membros desconhecem diretrizes da contrução civil necessárias ao projeto} & \parbox[t]{4cm}{Projeto escohido envolve estruturação de um prédio}                                       \\ \hline
  \end{tabular}
\end{table}

\subsection{Riscos Positivos}
\begin{table}[h]
  \centering
  \caption{Registro Riscos Positivos}
  \begin{tabular}{|l|c|l|l|}
    \hline
    \multicolumn{4}{|c|}{\textbf{Riscos Positivos}}                                                                                                                                             \\ \hline
    \multicolumn{1}{|c|}{\textbf{Evento}}  & \textbf{Identificador} & \multicolumn{1}{c|}{\textbf{Impacto}}                    & \multicolumn{1}{c|}{\textbf{Causa}}                            \\ \hline
    \parbox[t]{4cm}{Comunicação eficinete}                  & RP01                   & \parbox[t]{4.5cm}{Evita retrabalho e permite execução de tarefas coerentes} & \parbox[t]{4.5cm}{Boas ferramentas de comunicação e boa interação entre a equipe} \\ \hline
    \parbox[t]{4cm}{Conhecimentos da equipe bem nivelados.} & RP02                   & \parbox[t]{4.5cm}{Facilidade de Gerenciamento da equipe}                    & \parbox[t]{4.5cm}{Reuniões e discussões semanais}                                 \\ \hline
    \parbox[t]{4cm}{Equipes bem divididas.}                 & RP03                   & \parbox[t]{4.5cm}{Facilidade em gerenciar grupos separadamente}             & \parbox[t]{4.5cm}{Planejamento eficiente}                                         \\ \hline
  \end{tabular}
\end{table}

\pagebreak

\section{Definição de Probabilidade e Impacto dos Riscos}
Para realizar a análise dos riscos foram utilizados valores pré estabelecidos para as probabilidades e impactos.
\subsection{Probabilidade}
\begin{table}[h]
  \centering
  \caption{Probabilidade dos Riscos}
  \begin{tabular}{|c|c|c|}
    \hline
    \textbf{Probabilidade} & \textbf{Intervalo} & \textbf{Peso} \\ \hline
    Muito Baixa            & Menor que 20\%     & 1             \\ \hline
    Baixa                  & De 21\% a 40\%     & 2             \\ \hline
    Moderada               & De 41\% a 60\%     & 3             \\ \hline
    Alta                   & De 61\% a 80\%     & 4             \\ \hline
    Muito Alta             & Acima de 80\%      & 5             \\ \hline
  \end{tabular}
\end{table}

\subsection{Impacto}
\begin{table}[h]
  \centering
  \caption{Impacto dos Riscos}
  \label{my-label}
  \begin{tabular}{|c|c|c|}
    \hline
    \textbf{Impacto} & \textbf{Descrição}                              & \textbf{Representação} \\ \hline
    Muito Baixo      & Impacto é quase imperceptível ao projeto        & 1                      \\ \hline
    Baixo            & Pouco impacto no desenvolvimento do projeto     & 2                      \\ \hline
    Moderado         & Há um impacto grande porém recuperável          & 3                      \\ \hline
    Alto             & Há grande impacto no desenvolvimento do projeto & 4                      \\ \hline
    Muito Alto       & O impacto inviabiliza o projeto                 & 5                      \\ \hline
  \end{tabular}
\end{table}

\subsection{Definindo Prioridade}
A prioridade foi definida seguindo a tabela abaixo, que relaciona as probabilidades e os impactos.

\begin{table}[h]
\centering
\caption{Relação Probabilidade e Impacto dos Riscos}
\begin{tabular}{|c|c|c|c|}
\hline
\textbf{Prob./Imp.} & \textbf{Muito Baixo} & \textbf{Baixo} & \textbf{Moderado} \\ \hline
Muito Baixa         & 1                    & 2              & 3                 \\ \hline
Baixa               & 2                    & 4              & 6                 \\ \hline
Moderada            & 3                    & 6              & 9                 \\ \hline
Alta                & 4                    & 8              & 12                \\ \hline
Muito Alta          & 5                    & 10             & 15                \\ \hline
\end{tabular}
\end{table}

\begin{table}[h]
\centering
\caption{Definição de Prioridade de Riscos}
\label{my-label}
\begin{tabular}{|c|c|}
\hline
\textbf{Prioridade} & \textbf{Intervalo} \\ \hline
Baixa               & 1-5                \\ \hline
Média               & 6-15               \\ \hline
Alta                & 16-25              \\ \hline
\end{tabular}
\end{table}

\pagebreak


\section{Técnicas para o Planejamento de Resposta ao Risco}
\subsection{Definição das Técnicas}
De acordo com \cite{pmbok}, existem três estratégias que tipicamente lidam com ameaças ou riscos que podem ter impactos negativos nos objetivos do projeto, são elas:     prevenir, transferir e mitigar. Há, entretanto, uma quarta estratégia, aceitar,que pode ser usada tanto para riscos negativos ou ameaças quanto para riscos positivos ou oportunidades.

\begin{table}[h]
\centering
\caption{Técnicas para Resposta ao Riscos}
\label{my-label}
\begin{tabular}{|c|c|}
\hline
\textbf{Técnica} & \textbf{Descrição}                                                                                                                            \\ \hline
Evitar           & \parbox[t]{7cm}{Essa técnica envolve alterar o plano de gerenciamento do projeto para eliminar a ameaça totalmente.}                                           \\ \hline
Transferir       & \parbox[t]{7cm}{Essa técnica consiste em transferir o risco, e da resposta associada, para outro núcleo do projeto. Ressalta-se que o risco não é eliminado.} \\ \hline
Mitigar          & \parbox[t]{7cm}{Essa técnica consiste na redução da probabilidade e/ou do impacto de um risco para dentro de limites aceitáveis.}                              \\ \hline
\multicolumn{1}{|l|}{Aceitar} & \multicolumn{1}{l|}{\parbox[t]{7cm}{Essa técnica consiste em reconhecer a existência do risco e não agir, a menos que o risco ocorra.}}                        \\ \hline
\end{tabular}
\end{table}

\section{Análise, Identificação e Respostas aos Riscos}

\begin{table}[!h]
  \centering
  \caption{Análise, Identificação e Respostas aos Riscos}
  \begin{tabular}{|c|c|c|l|}
    \hline
    \multicolumn{4}{|c|}{\textbf{Riscos Negativos}}                                                                                                                                                                                               \\ \hline
    \textbf{Identificador} & \textbf{Probabilidade} & \textbf{Impacto} & \multicolumn{1}{c|}{\textbf{Ação}}                                                                                                                                       \\ \hline
    R01                    & Muito Baixa            & Alto             & \parbox[t]{5cm}{Aceitar - Continuar o trabalho e adequar a equipe as novas condições juntamente com uma tentativa de negociação com o cliente}                                            \\ \hline
    R02                    & Moderada               & Médio            & \parbox[t]{5cm}{Mitigar - Apoiar membros para evitar deistencia e dividir o trabalho para evitar sobrecarga}                                                                              \\ \hline
    R03                    & Muito Alta             & Muito Alto       & \parbox[t]{5cm}{Aceitar - Aguardar momento econômico mais oportuno para realização de proposta}                                                                                           \\ \hline
    R04                    & Moderada               & Médio            & \parbox[t]{5cm}{Evitar - Combrança maior por parte dos líderes motivando a equipe e em caso de desistencia realocar atividades e refazer planejamento}                                    \\ \hline
    R05                    & Moderada               & Médio            & \parbox[t]{5cm}{Mitigar - Manter um planejamento coerente com base em pesquisas de outras fontes para manter o planejamento coerente e replanejar caso o problema já tenha ocorrido}      \\ \hline
    R06                    & Baixo                  & Muito Alto       & \parbox[t]{5cm}{Aceitar - Manter contato constante com o cliente para que o mesmo não se surpreenda negativamente com o resultado final}                                                  \\ \hline
    R07                    & Baixo                  & Alto             & \parbox[t]{5cm}{Aceitar - Fazer calendário de atividades e entregas cumprindo-o durante todo o projeto}                                                                                   \\ \hline
    R08                    & Muito Alta             & Alto             & \parbox[t]{5cm}{Mitigar - Pesquisar sobre gerencia de grandes projetos e pedir ajuda aos professores para evitar ociosidade e sobrecarga}                                                 \\ \hline
    R09                    & Alto                   & Alto             & \parbox[t]{5cm}{Evitar - Perguntar aos professores sobre os entregáveis e pesquisar quais são os principais artefatos de gerência de projetos utilizados}                                 \\ \hline
    R10                    & Muito Alta             & Alto             & \parbox[t]{5cm}{Mitigar - Conseguir informações externas com professores, livros e sites para entender melhor o funcionamento da área e evitar problemas no projeto} \\ \hline
  \end{tabular}
\end{table}

\chapter{EAP}
A Estrutura Analítica de Projetos (EAP), é uma ferramenta visual que é feita a partir da decomposição das etapas do projeto em ordem cronológica. Ela funciona como um facilitador para a identificação de cada etapa do projeto, facilita os processos de gerenciamento e entregas bem como a estimativa de esforço, custo e duração do mesmo. Além da principal função, a definição do escopo do projeto. A EAP é representada em diagrama, começando do tópico mais geral, em seguida as principais etapas, e por fim as entregas que cada etapa necessita.

Para o projeto do Prédio Inteligente elaborou-se uma EAP para que fosse mais fácil a visualização das etapas que devem ser seguidas, além da definição do escopo do mesmo. Essa ferramenta também serve para que todos da equipe tenham acesso à modo que o projeto será desenvolvido. Sendo assim, as fases principais foram divididas em: Planejamento, Justificativa das Soluções e Viabilidade Econômica. Implicitamente estas fases representam os Pontos de Controle 1, 2 e 3 respectivamente. Consequentemente, foram definidas as entregas que devem ser feitas para cada Ponto de Controle.
 \begin{figure}[!h]
 	\centering
 	\includegraphics[keepaspectratio=true,scale=0.37]{figuras/eap.eps}
 	\caption{EAP do Projeto Prédio Inteligente}
 	\label{fig01}
 \end{figure}
