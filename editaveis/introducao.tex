\part{Aspectos Gerais}
\chapter[Introdução]{Introdução}

A Faculdade do Gama (FGA) é um dos campi presentes na Universade de Brasília. O
foco deste campus são as engenharias mais modernas, e portanto, mais
tecnológicas. Dentre os cursos presentes encontram-se as seguintes engenharias:
Aeroespacial, Automotiva, Energia, Eletrônica e Software.

 Apesar da proposta moderna e tecnológica, a realidade é bem diferente. O campus
 enfrenta problemas básicos de infraestrutura. Um exemplo disso é o planejamento
 ineficiente das salas: o sistema de iluminação é falho, pois mesmo durante do
 dia a maior parte das salas precisa permanecer com as luzes acessas; Além disso, tem-se: o
 posicionamento das carteiras e quadro torna ruim a visibilidade do conteúdo
 ministrado, o que é agravado pela iluminação (artificial e natural) que reflete
 no quadro; a poeira proveniente do estacionamento e arredores, a qual dificulta
 a implementação de equipamentos mais modernos, e a falta de climatização
 adequada que trás desconforto aos usuários.

 Além disso, há um desperdício de energia e de potencial energético na faculdade,
tendo em vista a forma como os equipamentos eletrônicos com os computadores
ficam sempre ligados, uso das luzes, como mencionado anteriormente, entre outras
práticas. A respeito do potencial energético, se dá pelo fato da faculdade não
utilizar o espaço que possui para implementar algum mecanismo de captação
energética. É notável também que há uma falta de sincronia entre a parte
administrativa e os usuários a respeito da reserva das salas e também pela parte
administrativa em relação ao recursos que necessitam de reparos, o que dá muito
provavelmente pela maneira ineficaz que os mesmos são feitos.

Por fim, há uma deficiencia no controle de quem entra e quem sai da
universidade, o que compromete a segurança dos usuários, bem como o excessivo
números de fraudes nas listas de chamadas nas disciplinas.

 O projeto propõe a solução dos diversos de problemas presentes
 na estrutura atual da faculdade. Assim, para a realização do mesmo, houve uma
 busca para estabelecer os riscos e elicitar meios para evitá-los e mitigá-los.
 Ademais, estruturou-se o projeto buscando estabelecer equipamentos, custos e
 meios mais aprofundados, conclusivos e detalhados de atender às demandas
 geradas pelos requisitos

\chapter{Escopo}
\section{Introdução}

Um ponto importante para a realização do projeto Prédio inteligente foi definir as
especificações-limite que o projeto contempla. Para tal fim utilizou-se a técnica 5W2H.

Dessa forma o escopo definido inclui a construção de um novo prédio na FGA, porém,
sem nenhuma modificação nos prédios já existentes. Ressalta-se entretanto, que algumas das soluções
aqui apresentadas tem potencial para serem utilizadas nos prédios existentes, caso a Universidade
de Brasília assim deseje, mas não faz parte do escopo deste trabalho definir isto.

Este projeto foi dividido em cinco frentes principais, de acordo com suas áreas de atuação:
\begin{itemize}
  \item \textbf{Estrutura e Materiais:} Responsável pela escolha e aplicação de materiais inteligentes na estrutura,
  otimização da planta e organização do espaço interno.

  \item \textbf{Smart Grid:} Responsável pela a  escolha e dimensionamento, de forma eficiente, do tipo de energia
  a ser utilizada para suprir a demanda energética da FGA.

  \item \textbf{Automação 1 - Controle de Acesso:} Responsável pelo controle de acesso à universidade, salas e
  laboratórios, além do registro de frequência durante as aulas.

  \item \textbf{Automação 2 - Instrumentação e Controle:} 	Responsável pela implantação dos dispositivos de
  instrumentação e controle do prédio. Isto engloba sensores, equipamentos de automação das salas e os
  equipamentos de controle e processamento destes dados.

  \item \textbf{Interfaces e Processamento de Software:} Responsável pelo processamento de todos os dados fornecidos
  pelos grupos anteriores, bem como sua disponibilização aos usuários por meio de interfaces gráficas.
\end{itemize}

As soluções para cada parte do projeto são abordadas a seguir.

\section{Soluções}

Na parte estrutural do projeto serão aplicados materiais inteligentes e
sustentáveis para amenizar a produção de resíduos e, consequentemente, o impacto
no meio ambiente. Além disso, serão consideradas algumas adaptações sobre o uso
das salas para se possa receber o sistema de automação e também definir que espaços serão
destinados a laboratórios ou salas de aula, de acordo com as necessidades dos
usuários. Por fim, haverá a alteração da posição dos elementos usados em sala
para melhorar o impacto que os mesmos têm no aprendizado.

Para a produção energética foram consideradas duas formas de suprir a demanda
energética da FGA: a primeira e principal é a geração de energia por meio de
placas fotovoltaicas, e a segunda, que será utilizada como fonte reserva, será
o suprimento por meio de um gerador movido a biodiesel, ativado caso haja algum
problema ou manuntenção nas placas fotovotaicas.

O projeto propõe o controle do acesso à Faculdade. Levando em conta que se trata de uma
universidade pública, não pode-se fechar completamente o acesso,
entretanto o objetivo é selecionar quem terá ou não acesso a partes
específicas dentro do novo prédio.

Inicialmente a ideia é restringir uma parte do estacionamento só para alunos
matriculados na faculdade, o acesso também será restringido nas salas e laboratórios com trancas que serão acopladas
nas portas. No caso de laboratórios e salas com equipamentos mais sofisticados,
somente será autorizada a entrada com um professor como responsável pelo uso do
ambiente. Em relação a salas simples, também haverá necessidade de um
responsável, contudo, poderá ser tanto aluno, monitor ou professor.

O controle de frequência também será averiguado em um aparelho eletrônico portátil
que cada professor possuirá. Essa tecnologia e modo de segurança já é utilizado
em várias universidades do mundo e especialmente em algumas instituições privadas
em Brasília. Esta medida será implementada dentro da UnB
principalmente por questões de segurança.

Tendo que até 7 mil pessoas, aproximadamente, podem utilizar o prédio, temos que o controle de
acesso ao prédio custará em média R\$ 19.000,00, considerando o preço de
carteirinhas e aparelhos de controle de acesso nas salas e no estacionamento. Os
custos relacionando a interface e processamento de dados estão atrelados a
quanto custará o desenvolvimento do aplicativo e a incrementação no banco de
dados para armazenar e processar os novos dados, com isso temos que em média
será gasto R\$ 70.000,00. A instrumentação e controle envolve aparelhos e sensores que vão identificar
condições do prédio e realizar ambientação nas salas custando aproximadamente
R\$ 65.000,00. As estruturas e materiais do prédio custarão por volta de
R\$ 1.050.000,00.

O controle energético do prédio é feito de forma inteligente e esse controle e
as diferentes formas de geração de alternativas de energia custarão
aproximadamente R\$ 1.000.000,00, totalizando para todo prédio um valor
próximo de R\$ 3.000.000,00 (3 milhões) de reais. Ressalta-se, porém, que este
valor não inclui quaisquer gastos com mão de obra, construção, etc, pois isto
não faz parte do escopo do projeto.

\chapter{Requisitos\label{ch:requisitos}}
\section{Backlog do Produto}
Com base nas técnicas apresentadas no Ponto de Controle 01 (Brainstorming, Entrevista), levantou-se os requisitos do projeto.

Estes requisitos foram documentados no Backlog do Produto, mostrado na figura ~\ref{fig:backlog}, onde os requisitos foram agrupados visando a rastreabilidade vertical, partindo de grandes blocos mais genéricos chamados Épicos, que são divididos em partes menores chamadas Features.

\begin{figure}[!h]
  \centering
  	\includegraphics[width=0.9\textwidth]{figuras/backlog.eps}
   \caption{Backlog do Produto\label{fig:backlog}}
\end{figure}

Os requisitos estão explicitados detalhadamente no apêndice \ref{appendix:apdcA}.

\chapter{Estudo de Riscos}
\section{Introdução}
Segundo o \cite{pmbok}, a Gerência de Riscos de Projeto inclui os processos de planejamento, identificação,
análise, planejamento de respostas e controle de riscos de um projeto. Essa gerência tem como objetivos aumentar a
probabilidade e o impacto dos eventos positivos, e reduzir a probabilidade e o impacto dos eventos negativos no projeto.

\section{Identificação dos Riscos}
Para a identificação dos riscos foi utilizada a técnica SWOT. Nesta técnica são utilizados 4 campos: Força (\textit{Strengths}),
 Oportunidades (\textit{Opportunities}), Fraqueza (\textit{Weaknesses}) e Ameaças (\textit{Threats}). Os campos Força e Oportunidade permitem uma
 visualização amplificada do projeto ao passo que as Fraquezas e Ameaças devem ser devidamente tratadas por meio do
 gerenciamento de riscos para manter o projeto seguindo corretamente. Os campos são definidos como:

\begin{itemize}
  \item \textbf{Força (\textit{Strengths}):} São elementos internos que representam benefícios para o projeto.
  \item \textbf{Oportunidades (\textit{Opportunities}):} São elementos externos que representam benefícios para o projeto.
  \item \textbf{Fraqueza (\textit{Weaknesses}):} São elementos internos que trazem prejuízos para o projeto.
  \item \textbf{Ameaças (\textit{Threats}):} São elementos externos que trazem prejuízos para o projeto.
\end{itemize}

Segue-se a representação gráfica desta técnica aplicada à este projeto.

\pagebreak

\begin{figure}[!h]
 \centering
 \includegraphics[keepaspectratio=true,scale=0.23]{figuras/swot.eps}
 \caption{Técnica SWOT}
\end{figure}

\section{Categorização dos Riscos}
Risco é um evento ou uma condição incerta que, se ocorrer, provocará um efeito positivo ou negativo nos objetivos do projeto tais como custo, escopo, prazo ou qualidade \cite{bianco}. A Estrutura Analítica de Riscos permite uma organização onde há maior compreensão, gerenciamento e comunicação pois apresenta seus resultados de forma estruturada como mostrado abaixo.
\subsection{Descrição dos Itens da EAR}
\begin{itemize}
  \item Técnico
    \begin{itemize}
      \item Inexperiência em construção civil: Diz respeito a falta de familiaridade do grupo com técnicas pautadas na engenharia civil.
    \end{itemize}
  \item Externo
  \begin{itemize}
    \item \textbf{Mudanças no calendário acadêmico:} Diz respeito ao ambiente acadêmico onde está inserido o projeto e os riscos de mudança no calendário.
    \item \textbf{Ausência de informações quanto a documentação a ser elaborada:} Diz respeito a falta de informações na disciplina quanto ao que deve ser feito em que etapa.
    \item \textbf{Crise econômica:} Diz respeito a instabilidade econômica pela qual vem passando o país nos últimos anos.
    \item \textbf{Resultado final não atender as expectativas:} Diz respeito a não aceitação do projeto por parte do cliente e stakeholders.
  \end{itemize}


  \item Organizacional
  \begin{itemize}
    \item \textbf{Atraso na entrega das atividades:} Diz respeito a desorganização da equipe quanto a respeitar o calendário.
    \item \textbf{Redução dos membros:} Diz respeito aos riscos relacionados aos recursos humanos do projeto, com membros que saem da equipe.
    \item \textbf{Falta de comprometimento dos membros:} Diz respeito a problemas de gerenciamento e cobrança para com membros ociosos.
    \item \textbf{Erros de estimativa e previsões:} Diz respeito a falta de experiência da equipe quanto a gerência de projetos que pode acarretar em atrasos ou adiantamentos.
  \end{itemize}

  \item Gerência
  \begin{itemize}
    \item \textbf{Inexperiência em projetos grandes:} Diz respeito a imaturidade da equipe ao lidar com uma equipe grande.
  \end{itemize}

\end{itemize}

\pagebreak

\section{Registro de Riscos Identificados}
\subsection{Riscos Negativos}
\begin{table}[h]
  \centering
  \caption{Registro dos Riscos Negativos}
  \begin{tabular}{|l|c|l|l|}
    \hline
    \multicolumn{4}{|c|}{\textbf{Riscos Negativos}}                                                                                                                                                                                                               \\ \hline
    \multicolumn{1}{|c|}{\textbf{Evento}}                          & \textbf{Identificador} & \multicolumn{1}{c|}{\textbf{Impacto}}                                    & \multicolumn{1}{c|}{\textbf{Causa}}                                                      \\ \hline
    \parbox[t]{5cm}{Mudança do calendário acadêmico.}                               & R01                    & \parbox[t]{4cm}{Adiantamento ou atrazo das entregas}                                      & \parbox[t]{4cm}{Greves ou qualquer outro empecilho relacionado a universidade}                            \\ \hline
    Redução dos membros.                                           & R02                    & \parbox[t]{4cm}{Sobrecarga de trabalho para os outros membros do grupo}                   & \parbox[t]{4cm}{Trancamento ou desistência da diciplina}                                                  \\ \hline
    \parbox[t]{5cm}{Crise econômica no país pode tornar o projeto inviável.}        & R03                    & \parbox[t]{4cm}{Projeto não será utilizado}                                               & \parbox[t]{4cm}{Atual instabilidade econômica do país}                                                    \\ \hline
    \parbox[t]{5cm}{Falta de comprometimento dos membros.}                          & R04                    & \parbox[t]{4cm}{Sobrecarga de trabalho para os outros membros do grupo}                   & \parbox[t]{4cm}{Membros querem desistir da diciplina}                                                     \\ \hline
    \parbox[t]{5cm}{Erros de estimativas e previsões.}                              & R05                    & \parbox[t]{4cm}{Previsões e estimativas muito foras da realidade do prédio}               & \parbox[t]{4cm}{Inexperiência nas atividades de estimativa e previsões}                                   \\ \hline
    \parbox[t]{5cm}{Resultado final não atender às expectativas.}                   & R06                    & \parbox[t]{4cm}{Rejeição do projeto}                                                      & \parbox[t]{4cm}{Falhas nas especificações e validações dos requisitos}                                    \\ \hline
    \parbox[t]{5cm}{Atraso nas entregas das atividades.}                            & R07                    & \parbox[t]{4cm}{Atraso no cronograma}                                                     & \parbox[t]{4cm}{Falta de motivação experiência ou planejamento da equipe}                                 \\ \hline
    \parbox[t]{5cm}{Inexperiência da equipe em projetos grandes.}                   & R08                    & \parbox[t]{4cm}{Problemas de gerência de projeto}                                         & \parbox[t]{4cm}{Membros nunca trabalharam em um grande projeto}                                           \\ \hline
    \parbox[t]{5cm}{Ausência de informações quanto à documentação a ser elaborada.} & R09                    & \parbox[t]{4cm}{Documentação entregue incompleta}                                         & \parbox[t]{4cm}{Falta de fontes de informação sobre a necessidade de documentos e informações no projeto} \\ \hline
    \parbox[t]{5cm}{Inexperiência quanto à construção civil.}                       & R10                    & \parbox[t]{4cm}{Membros desconhecem diretrizes da contrução civil necessárias ao projeto} & \parbox[t]{4cm}{Projeto escohido envolve estruturação de um prédio}                                       \\ \hline
  \end{tabular}
\end{table}

\subsection{Riscos Positivos}
\begin{table}[h]
  \centering
  \caption{Registro Riscos Positivos}
  \begin{tabular}{|l|c|l|l|}
    \hline
    \multicolumn{4}{|c|}{\textbf{Riscos Positivos}}                                                                                                                                             \\ \hline
    \multicolumn{1}{|c|}{\textbf{Evento}}  & \textbf{Identificador} & \multicolumn{1}{c|}{\textbf{Impacto}}                    & \multicolumn{1}{c|}{\textbf{Causa}}                            \\ \hline
    \parbox[t]{4cm}{Comunicação eficinete}                  & RP01                   & \parbox[t]{4.5cm}{Evita retrabalho e permite execução de tarefas coerentes} & \parbox[t]{4.5cm}{Boas ferramentas de comunicação e boa interação entre a equipe} \\ \hline
    \parbox[t]{4cm}{Conhecimentos da equipe bem nivelados.} & RP02                   & \parbox[t]{4.5cm}{Facilidade de Gerenciamento da equipe}                    & \parbox[t]{4.5cm}{Reuniões e discussões semanais}                                 \\ \hline
    \parbox[t]{4cm}{Equipes bem divididas.}                 & RP03                   & \parbox[t]{4.5cm}{Facilidade em gerenciar grupos separadamente}             & \parbox[t]{4.5cm}{Planejamento eficiente}                                         \\ \hline
  \end{tabular}
\end{table}

\pagebreak

\section{Definição de Probabilidade e Impacto dos Riscos}
Para realizar a análise dos riscos foram utilizados valores pré estabelecidos para as probabilidades e impactos.
\subsection{Probabilidade}
\begin{table}[h]
  \centering
  \caption{Probabilidade dos Riscos}
  \begin{tabular}{|c|c|c|}
    \hline
    \textbf{Probabilidade} & \textbf{Intervalo} & \textbf{Peso} \\ \hline
    Muito Baixa            & Menor que 20\%     & 1             \\ \hline
    Baixa                  & De 21\% a 40\%     & 2             \\ \hline
    Moderada               & De 41\% a 60\%     & 3             \\ \hline
    Alta                   & De 61\% a 80\%     & 4             \\ \hline
    Muito Alta             & Acima de 80\%      & 5             \\ \hline
  \end{tabular}
\end{table}

\subsection{Impacto}
\begin{table}[h]
  \centering
  \caption{Impacto dos Riscos}
  \label{my-label}
  \begin{tabular}{|c|c|c|}
    \hline
    \textbf{Impacto} & \textbf{Descrição}                              & \textbf{Representação} \\ \hline
    Muito Baixo      & Impacto é quase imperceptível ao projeto        & 1                      \\ \hline
    Baixo            & Pouco impacto no desenvolvimento do projeto     & 2                      \\ \hline
    Moderado         & Há um impacto grande porém recuperável          & 3                      \\ \hline
    Alto             & Há grande impacto no desenvolvimento do projeto & 4                      \\ \hline
    Muito Alto       & O impacto inviabiliza o projeto                 & 5                      \\ \hline
  \end{tabular}
\end{table}

\subsection{Definindo Prioridade}
A prioridade foi definida seguindo a tabela abaixo, que relaciona as probabilidades e os impactos.

\begin{table}[h]
\centering
\caption{Relação Probabilidade e Impacto dos Riscos}
\begin{tabular}{|c|c|c|c|}
\hline
\textbf{Prob./Imp.} & \textbf{Muito Baixo} & \textbf{Baixo} & \textbf{Moderado} \\ \hline
Muito Baixa         & 1                    & 2              & 3                 \\ \hline
Baixa               & 2                    & 4              & 6                 \\ \hline
Moderada            & 3                    & 6              & 9                 \\ \hline
Alta                & 4                    & 8              & 12                \\ \hline
Muito Alta          & 5                    & 10             & 15                \\ \hline
\end{tabular}
\end{table}

\begin{table}[h]
\centering
\caption{Definição de Prioridade de Riscos}
\label{my-label}
\begin{tabular}{|c|c|}
\hline
\textbf{Prioridade} & \textbf{Intervalo} \\ \hline
Baixa               & 1-5                \\ \hline
Média               & 6-15               \\ \hline
Alta                & 16-25              \\ \hline
\end{tabular}
\end{table}

\pagebreak


\section{Técnicas para o Planejamento de Resposta ao Risco}
\subsection{Definição das Técnicas}
De acordo com \cite{pmbok}, existem três estratégias que tipicamente lidam com ameaças ou riscos que podem ter impactos negativos nos objetivos do projeto, são elas:     prevenir, transferir e mitigar. Há, entretanto, uma quarta estratégia, aceitar,que pode ser usada tanto para riscos negativos ou ameaças quanto para riscos positivos ou oportunidades.

\begin{table}[h]
\centering
\caption{Técnicas para Resposta ao Riscos}
\label{my-label}
\begin{tabular}{|c|c|}
\hline
\textbf{Técnica} & \textbf{Descrição}                                                                                                                            \\ \hline
Evitar           & \parbox[t]{7cm}{Essa técnica envolve alterar o plano de gerenciamento do projeto para eliminar a ameaça totalmente.}                                           \\ \hline
Transferir       & \parbox[t]{7cm}{Essa técnica consiste em transferir o risco, e da resposta associada, para outro núcleo do projeto. Ressalta-se que o risco não é eliminado.} \\ \hline
Mitigar          & \parbox[t]{7cm}{Essa técnica consiste na redução da probabilidade e/ou do impacto de um risco para dentro de limites aceitáveis.}                              \\ \hline
\multicolumn{1}{|l|}{Aceitar} & \multicolumn{1}{l|}{\parbox[t]{7cm}{Essa técnica consiste em reconhecer a existência do risco e não agir, a menos que o risco ocorra.}}                        \\ \hline
\end{tabular}
\end{table}

\section{Análise, Identificação e Respostas aos Riscos}

\begin{table}[!h]
  \centering
  \caption{Análise, Identificação e Respostas aos Riscos}
  \begin{tabular}{|c|c|c|l|}
    \hline
    \multicolumn{4}{|c|}{\textbf{Riscos Negativos}}                                                                                                                                                                                               \\ \hline
    \textbf{Identificador} & \textbf{Probabilidade} & \textbf{Impacto} & \multicolumn{1}{c|}{\textbf{Ação}}                                                                                                                                       \\ \hline
    R01                    & Muito Baixa            & Alto             & \parbox[t]{5cm}{Aceitar - Continuar o trabalho e adequar a equipe as novas condições juntamente com uma tentativa de negociação com o cliente}                                            \\ \hline
    R02                    & Moderada               & Médio            & \parbox[t]{5cm}{Mitigar - Apoiar membros para evitar deistencia e dividir o trabalho para evitar sobrecarga}                                                                              \\ \hline
    R03                    & Muito Alta             & Muito Alto       & \parbox[t]{5cm}{Aceitar - Aguardar momento econômico mais oportuno para realização de proposta}                                                                                           \\ \hline
    R04                    & Moderada               & Médio            & \parbox[t]{5cm}{Evitar - Combrança maior por parte dos líderes motivando a equipe e em caso de desistencia realocar atividades e refazer planejamento}                                    \\ \hline
    R05                    & Moderada               & Médio            & \parbox[t]{5cm}{Mitigar - Manter um planejamento coerente com base em pesquisas de outras fontes para manter o planejamento coerente e replanejar caso o problema já tenha ocorrido}      \\ \hline
    R06                    & Baixo                  & Muito Alto       & \parbox[t]{5cm}{Aceitar - Manter contato constante com o cliente para que o mesmo não se surpreenda negativamente com o resultado final}                                                  \\ \hline
    R07                    & Baixo                  & Alto             & \parbox[t]{5cm}{Aceitar - Fazer calendário de atividades e entregas cumprindo-o durante todo o projeto}                                                                                   \\ \hline
    R08                    & Muito Alta             & Alto             & \parbox[t]{5cm}{Mitigar - Pesquisar sobre gerencia de grandes projetos e pedir ajuda aos professores para evitar ociosidade e sobrecarga}                                                 \\ \hline
    R09                    & Alto                   & Alto             & \parbox[t]{5cm}{Evitar - Perguntar aos professores sobre os entregáveis e pesquisar quais são os principais artefatos de gerência de projetos utilizados}                                 \\ \hline
    R10                    & Muito Alta             & Alto             & \parbox[t]{5cm}{Mitigar - Conseguir informações externas com professores, livros e sites para entender melhor o funcionamento da área e evitar problemas no projeto} \\ \hline
  \end{tabular}
\end{table}

\chapter{EAP}
A Estrutura Analítica de Projetos (EAP), é uma ferramenta visual que é feita a partir da decomposição das etapas do projeto em ordem cronológica. Ela funciona como um facilitador para a identificação de cada etapa do projeto, facilita os processos de gerenciamento e entregas bem como a estimativa de esforço, custo e duração do mesmo. Além da principal função, a definição do escopo do projeto. A EAP é representada em diagrama, começando do tópico mais geral, em seguida as principais etapas, e por fim as entregas que cada etapa necessita.

Para o projeto do Prédio Inteligente elaborou-se uma EAP para que fosse mais fácil a visualização das etapas que devem ser seguidas, além da definição do escopo do mesmo. Essa ferramenta também serve para que todos da equipe tenham acesso à modo que o projeto será desenvolvido. Sendo assim, as fases principais foram divididas em: Planejamento, Justificativa das Soluções e Viabilidade Econômica. Implicitamente estas fases representam os Pontos de Controle 1, 2 e 3 respectivamente. Consequentemente, foram definidas as entregas que devem ser feitas para cada Ponto de Controle.
 \begin{figure}[!h]
 	\centering
 	\includegraphics[keepaspectratio=true,scale=0.37]{figuras/eap.eps}
 	\caption{EAP do Projeto Prédio Inteligente}
 	\label{fig01}
 \end{figure}
