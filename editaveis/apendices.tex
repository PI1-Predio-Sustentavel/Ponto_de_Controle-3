\begin{apendicesenv}

\partapendices
\chapter{Backlog do Projeto}
\label{appendix:apdcA}

Segue-se as especificações do projeto. Estas estão divididas conforme as áreas do projeto. Ressalta-se que
 para detalhes mais específicos deve-se obervar os tópicos presentes na Parte \ref{sol}.

\section{Estruturas e Materiais}
  \begin{itemize}
    \item Otimização da planta:
    \begin{itemize}
      \item Definição das funcionalidades dos espaços
      \begin{itemize}
      \item 1 Lanchonete
      \item 1 Xerox
      \item 1 Sala de controle
      \item 2 Salas de computadores
      \item 11 Salas de aulas
      \item 9 Laboratórios
      \item 1 Sala de estudos
      \end{itemize}
      \item Posicionamento do prédio
      \begin{itemize}
      \item Uso da carta solar para alívio térmico
      \begin{itemize}
      \item Salas de aula voltadas para leste
      \end{itemize}
      \end{itemize}
      \item Soluções para poeira
      \begin{itemize}
      \item Brises
      \item Jardins Verticais
      \item Arborização
      \end{itemize}
      \item Soluções para climatização
      \begin{itemize}
      \item Ar-condicionado
      \end{itemize}
    \end{itemize}
    \item Organização interna:
  \begin{itemize}
  \item Disposição dos equipamentos na sala para melhorar a interação dos alunos com os mesmos
    \item Otimização da sala
    \begin{itemize}
    \item Quadros panorâmico
  \item Tablados de madeira
  \item Tomadas distribuídas
  \item Caixas de som
    \end{itemize}
  \end{itemize}
    \item Materiais sustentáveis:
    \begin{itemize}
    \item Cimento de Alto Desempenho(CAD)
    \begin{itemize}
    \item Peso próprio reduzido
  \item Redução de custos
    \end{itemize}
    \item Estruturas de Aço
    \begin{itemize}
    \item Possibilidade de reciclagem
    \end{itemize}
  \item Esquadrias de PVC
    \begin{itemize}
    \item Não necessitam de manutenção
    \end{itemize}
    \item Tintas Ecológicas
    \begin{itemize}
    \item Não agridem o meio ambiente
    \end{itemize}
    \end{itemize}
  \end{itemize}
\section{Smart Grid}

  \begin{itemize}
    \item Tipos de energia a serem utilizados
    \begin{itemize}
      \item Solar
        \begin{itemize}
          \item Normas CEB
          \item 440 placas necessárias 
          \item Placas posicionadas no telhado
          \item Modelos das Placas - 260w
          \item 5 Inversores de 20Kw
          \item Estimativa energética 23.000kwh/mês
          \item Dimensionamento do sistema fotovoltaico 
          \begin{itemize}
          \item 440 placas
      \item 5 Inversores
        \item 5 string box
      \item Data Logger
      \item Medidor bidirecional 
      \item 5 multimedidores de energia
          \end{itemize}
        \end{itemize}
      \item Gerador movido a biodiesel
      \begin{itemize}
          \item Quantidade de litros de Biodiesel necessários: 73,5 litros 
      \item Relacionado ao Projeto  Biogama
          \end{itemize}
      \item Manutenções
      \begin{itemize}
      \item Equipamento que verifica a funcionalidades das placas: DATA LOGGER
    \item Verificação de quantidade energética consumida
    \item Verificação de quantidade energética gerada
    \item Reciclagem das placas
      \end{itemize}
    \end{itemize}
  \end{itemize}

\section{Interfaces e Processamento de Software}

  \begin{itemize}
    \item Aplicativo
      \begin{itemize}
        \item Informações sobre o consumo de energia
        \begin{itemize}
        \item Exibir consumo de energia
        \item Exibir quantidade de energia gerada localmente
      \end{itemize}
        \item Informações sobre salas
        \begin{itemize}
        \item Exibir lotação das salas
        \item Verificar disponibilidade da sala
        \item Reservar uma sala
      \end{itemize}
         
        \item Notificar defeitos em objetos
        \begin{itemize}
        \item Ler e enviar código presente nos objetos
      \end{itemize}
      \end{itemize}
     \item Processamento de dados
     \begin{itemize}
        \item Processar dados sobre energia
        \begin{itemize}
        \item Processar dados dos inversores e fazer estimativa de produção energética.
        \item Coletar produção e consumo de energia para disponibilizar no aplicativo
        \end{itemize}
        \item Processar dados sobre sensores
        \begin{itemize}
        \item Processar dados vindos dos sensores preventivos que detectam problemas no prédio
        \item Processar dados vindos dos sensores preventivos que detectam problemas no prédio
        \item Processar dados da ambientação da sala e proporcionar o controle das funções som, ar condicionado, lâmpada e persianas
        \end{itemize}
      \end{itemize}
     \item Banco de Dados
          \begin{itemize}
        \item Armazenar dados energéticos
        \begin{itemize}
        \item Geração e consumo de energia
      \end{itemize}
        \item Armazenar dados do controle de acesso
        \begin{itemize}
        \item Dados dos alunos e seus respectivos ids nas carteirinhas bem como um token gerado aleatoriamente.
        \end{itemize}
        \item Armazenar dados dos sensores para realizar um levantamento periódico das condições dos prédios
        \begin{itemize}
        \item Analisar sensores de calor, iluminação, movimento.
        \end{itemize}

      \end{itemize}
  \end{itemize}

\section{Controle de Acesso (Automação 1)}
  \begin{itemize}
    \item Sistema RFID
      \begin{itemize}
        \item Sistema semelhante ao usado em BRT’s e metrô.
        \item Mantêm a comunicação com todos os subgrupos de controle de acesso
      \end{itemize}
    \item Controle de Acesso no estacionamento privativo
      \begin{itemize}
        \item Apresentação da carteirinha no aparelho que lê as informações do chip do documento
        \item Comunicação dos dados da carteirinha com a cancela
      \end{itemize}
    \item  Controle de Entrada nas Salas e Laboratórios
      \begin{itemize}
        \item A fechadura é conectada com o aparelho que lê as informações do chip do documento e reconhece a pessoa como professor(a) ou como aluno responsável pela sala naquele período de tempo
        \item Controle de Acesso dos Alunos
        \begin{itemize}
        \item Alunos precisam pedir uma liberação dada pela secretaria do prédio
        \item O tempo dos alunos nas salas é limitado
        \item Um aluno especificamente será o responsável pela sala durante o horário estipulado
        \item Caso o aluno responsável precise se ausentar da sala, outro terá que se encaminhar à secretaria e transferir a responsabilidade da sala
        \end{itemize}
        \item Controle de Acesso dos Professores
        \begin{itemize}
        \item Todos professores podem entrar em qualquer sala ou laboratório e em qualquer horário
        \end{itemize}
      \end{itemize}
    \item Controle de frequência nas aulas
      \begin{itemize}
        \item Aparelho portátil que lê as informações do chip da carteirinha e reconhece o aluno por nome completo e matrícula
        \begin{itemize}
        \item O aparelho é propriedade de cada professor
        \end{itemize}
        \item Usar o código de barra da carteirinha para validar presença;
        \item No aparelho são arquivadas as matérias, com código, e horários de cada professor individualmente
        \item O funcionamento não depende de internet WiFi durante seu uso, entretanto, só com internet é possível armazenar todos os dados anteriormente inseridos
        \item Gera um token para cada carteirinha no banco de dados da universidade impedindo plágios
      \end{itemize}
  \end{itemize}
\section{Instrumentação e Controle (Automação 2)}
  \begin{itemize}
    \item Ambientação da Sala
      \begin{itemize}
        \item Persianas Inteligentes
        \begin{itemize}
          \item Regulagem de luz natural
            \item Controle de Poeira
            \item Isolamento Acústico
        \end{itemize}
        \item Ares-condicionados
        \begin{itemize}
          \item Climatização independente de cada sala
            \item Controle centralizado
        \end{itemize}
        \item Lâmpadas inteligentes
        \begin{itemize}
          \item Sensor de presença
            \item Regulagem de luz autônoma
            \item Controle centralizado
        \end{itemize}
        \item Sistema de Som
        \begin{itemize}
          \item Fácil acesso
            \item Melhoria da acústica
        \end{itemize}
      \end{itemize}
    \item Digitalização e Integração dos equipamentos para aula
      \begin{itemize}
        \item Mesa inteligente
        \begin{itemize}
          \item Acesso à internet
            \item Auxílio para o professor durante a aula
            \item Controle centralizado para equipamentos
            \item Conexão Bluetooth
            \item Conexão WiFi
            \item Conexão Infravermelho
        \end{itemize}
        \item Sistema de projetores
        \begin{itemize}
          \item Exibição de vídeos e slides durante as aulas
            \item Conexão com a Mesa Inteligente 
        \end{itemize}
      \end{itemize}
    \item Status da Sala
      \begin{itemize}
        \item Painel informativo
         \begin{itemize}
          \item Exibição de informações sobre o estado da sala
            \item Exibição de informações sobre aulas
            \item Exibição de informativos de assuntos diversos
        \end{itemize}
      \end{itemize}
    \item Monitoramento do Prédio
      \begin{itemize}
        \item Feedback do funcionamento e uso dos equipamentos eletrônicos\begin{itemize}
          \item Fornecimento de estatísticas de uso
            \item Monitoramento do funcionamento de equipamentos
            \item Manutenção preditiva
            \item Prevenção de falhas
        \end{itemize}
        
        
      \end{itemize}
  \end{itemize}

\end{apendicesenv}
